% vim: set foldmethod=marker:
\documentclass[12pt]{article}
%Packages{{{
\usepackage[margin=1in]{geometry} 
\usepackage{amsmath,amsthm,amssymb,mathtools}
\usepackage[makeroom]{cancel}
\usepackage{textcomp, graphicx, array, listings, soul,xspace} 
\usepackage{algorithmicx,algorithm, algpseudocode}
\usepackage[normalem]{ulem}
\usepackage{stackengine}
\usepackage{color} %red, green, blue, yellow, cyan, magenta, black, white
\usepackage{xparse, tikz, tikz-3dplot}
\usetikzlibrary{matrix,backgrounds, automata, arrows,fit,shapes,calc,positioning,intersections}
\usepackage{pgfplots}
\pgfplotsset{compat=1.12}
\pgfdeclarelayer{myback}
\pgfsetlayers{myback,background,main}\definecolor{mygreen}{RGB}{28,172,0} % color values Red, Green, Blue
\definecolor{mylilas}{RGB}{170,55,241}
\tdplotsetmaincoords{80}{120}
\DeclarePairedDelimiter\floor{\lfloor}{\rfloor}
\DeclarePairedDelimiter{\ceil}{\lceil}{\rceil}
\newcommand{\N}{\mathbb{N}}
\newcommand{\Z}{\mathbb{Z}}
\newcommand*\conj[1]{\bar{#1}}
\tikzset{mycolor/.style = {line width=1bp,color=#1}}%
\tikzset{myfillcolor/.style = {draw,fill=#1}}%
\NewDocumentCommand{\highlight}{O{blue!40} m m}{%
\draw[mycolor=#1] (#2.north west)rectangle (#3.south east);
}
\NewDocumentCommand{\fhighlight}{O{blue!40} m m}{%
\draw[myfillcolor=#1] (#2.north west)rectangle (#3.south east);
}
\newcommand*{\eg}{e.g.\@\xspace}
\newcommand*{\ie}{i.e.\@\xspace}

\makeatletter
\newcommand*{\etc}{%
    \@ifnextchar{.}%
        {etc}%
        {etc.\@\xspace}%
}
\makeatother
%}}}
%Environments{{{ 
\newenvironment{theorem}[2][Theorem]{\begin{trivlist}
\item[\hskip \labelsep {\bfseries #1}\hskip \labelsep {\bfseries #2.}]}{\end{trivlist}}
\newenvironment{lemma}[2][Lemma]{\begin{trivlist}
\item[\hskip \labelsep {\bfseries #1}\hskip \labelsep {\bfseries #2.}]}{\end{trivlist}}
\newenvironment{problem}[2][Problem]{\begin{trivlist}
\item[\hskip \labelsep {\bfseries #1}\hskip \labelsep {\bfseries #2.}]}{\end{trivlist}}
\newenvironment{claim}[2][Claim]{\begin{trivlist}
\item[\hskip \labelsep {\bfseries #1}\hskip \labelsep {\bfseries #2.}]}{\end{trivlist}}%}}}
\begin{document}
\title{EENG 348/CPSC338: Digital Systems}
\author{Kevin Truong \& Rob Brunstad} 
\maketitle
\section{Lab 5 Implementation}
\subsection{Part 1}

For creating processes on the fly, we traverse the ready queue whenever there 
is a call to ready\_queue and add it to the end of the queue. To free the memory
associated with the process\_t struct, we free the stack pointer and then the 
entire struct. 
% Part 1 Algorithm {{{
\begin{algorithm}[!h]
\caption{Memory}
\begin{algorithmic}[1]
    \Procedure{Freeing Memory}{}
    \State{free(process\_t$\rightarrow$sp)}
    \State{free(process\_t)}
    \EndProcedure{}
\end{algorithmic}
\end{algorithm}
% }}}

\subsection{Part 2}
Part 2 was implemented by keeping a shorted queue at all times. Process\_create\_prio()
was very similar to Procress\_create and only differed by adding a new field 
called prio into process\_t. 
% Part 2 Algorithm {{{
\begin{algorithm}[!h]
\caption{Create}
\begin{algorithmic}[1]
    \Procedure{Process\_Create\_Prio}{}
        \State{Allocate space for the stack pointer and check if the stack can allocate that much memory}
        \State{Malloc space for a new process}
        \State{Set the fields and assign prio to be equal to the input priority}
    \EndProcedure{}
\end{algorithmic}
\end{algorithm}
% }}}
It is also important to note the Process\_create() was modified so that by default,
ordinary processes have priority 128. The queue was sorting in decreasing order using the priority as the key. 
% Part 2 Algorithm {{{
\begin{algorithm}[!h]
\caption{Sorting}
\begin{algorithmic}[1]
    \Procedure{Sort Queue in Decreasing Order}{}
        \If{the tail has greater or equal priority than the new process}
            \State{Add the procress to the end}
        \ElsIf{the new procress has higher priority}
            \State{Add to the beginning of the queue}
        \Else{}
            \State{Traverse queue until the new process has higher priority than some process in the queue}
        \EndIf{}
    \EndProcedure{}
\end{algorithmic}
\end{algorithm}
% }}}

\subsection{Part 3}
For the real-time scheduling queue, we assigned the real-time task a priority between
129 and 255, since normal tasks have a priority of 128. We kept track of the shortest deadline and 
the longest deadline and we updated the deadlines of all the real-time tasks if either
the shortest or the longest deadlines change. This maintained the decreasing order
in the queue. To keep track of wrong worst case execution estimated, we kept track of 
the duration of the long each process ran. 

% Part 3 Algorithm {{{
\begin{algorithm}[!h]
\caption{Real-Time}
\begin{algorithmic}[1]
    \Procedure{Real-time Scheduling}{}
     \If{no real time tasks have been scheduled yet} 
        \State{new\_process$\rightarrow$prio = 255}
        \State{shortest deadline = deadline}
        \State{longest deadline = deadline}
    \ElsIf{there is a new process with a shorter or longer deadline} 
        \State{Recompute shortest and longest deadlines}
        \State{Recompute priority for all real-time jobs}
    \Else{}
        \State{Calculate priority with given shortest and longest deadlines}
    \EndIf{}
    \EndProcedure{}
\end{algorithmic}
\end{algorithm}
% }}}

\end{document}
