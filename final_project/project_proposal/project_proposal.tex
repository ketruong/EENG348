% vim: set foldmethod=marker:
\documentclass[12pt]{article}
% Packages{{{
\usepackage[margin=1in]{geometry} 
\usepackage{amsmath,amsthm,amssymb}
\usepackage{textcomp}
\usepackage[makeroom]{cancel}
\usepackage{hyperref}
\usepackage{mathtools}
\usepackage{listings}
\usepackage{soul}
\DeclarePairedDelimiter\floor{\lfloor}{\rfloor}
\DeclarePairedDelimiter{\ceil}{\lceil}{\rceil}
\newcommand{\N}{\mathbb{N}}
\newcommand{\Z}{\mathbb{Z}}

%}}}
%Environments{{{ 
\newenvironment{theorem}[2][Theorem]{\begin{trivlist}
\item[\hskip \labelsep {\bfseries #1}\hskip \labelsep {\bfseries #2.}]}{\end{trivlist}}
\newenvironment{lemma}[2][Lemma]{\begin{trivlist}
\item[\hskip \labelsep {\bfseries #1}\hskip \labelsep {\bfseries #2.}]}{\end{trivlist}}
\newenvironment{problem}[2][Problem]{\begin{trivlist}
\item[\hskip \labelsep {\bfseries #1}\hskip \labelsep {\bfseries #2.}]}{\end{trivlist}}
\newenvironment{claim}[2][Claim]{\begin{trivlist}
\item[\hskip \labelsep {\bfseries #1}\hskip \labelsep {\bfseries #2.}]}{\end{trivlist}}%}}}
\begin{document}
\title{EENG 348/CPSC\textsc{338}: Digital Systems}
\author{Kevin Truong \& Rob Brunstad} 
\maketitle
% Introduction {{{
\section{Introduction}
    Snake is a classic arcade game in which a player maneuvers his or her "snake" around some rectangular area, collecting food and avoiding collisions between the snake's head and any part of the snake's body. The snake advances at a constant rate, although the length of the snake's body increases as it consumes food. New instances of food appear as it is consumed such that there is always exactly one piece of food present on the playing area. Typically, a single player controls a single snake, and the goal is to maximize food consumption.
       \begin{figure}[h!]
        \centering
        \includegraphics[scale=0.5]{snake.png}
        \caption{Snake in the 1970's}\label{button}
    \end{figure}
We propose implementing a two-player version of snake on the Arduino. Food will appear in the usual way, and players will compete over it so that they may grow larger and therefore pose a larger threat to their opponents. Other gameplay elements will be added to increase complexity, including specially-colored "food" that grants a player's snake temporary invincibility.

We will display the playing area on a 16 x 32 RGB LED matrix panel, and players will control their respective snakes' movement by manipulating a PlayStation 2 Joystick.

% }}}
% Materials {{{
\section{Materials}
\begin{itemize}
    \item Arduino UNO
    \item 3D-printed housing
    \item \href{https://www.adafruit.com/product/420}{16x32 RGB RGB LED Matrix Panel}
    \item \href{https://www.newegg.com/Product/Product.aspx?Item=9SIABKS5R53998i&ignorebbr=1&nm_mc=KNC-GoogleMKP-PC&cm_mmc=KNC-GoogleMKP-PC-_-pla-_-Gadgets-_-9SIABKS5R53998&gclid=EAIaIQobChMIrq-a7qKL2gIVG0sNCh11XASxEAkYAiABEgKUXPD_BwE&gclsrc=aw.ds}{Playstation 2 Game Joystick Axis Sensor Module (x2)}
    
    \item \href{https://www.adafruit.com/product/327}{2.1mm Barrel Jack Extension Cable}
    \item \href{https://www.adafruit.com/product/276}{5V 2A Switching Power Supply}
    \item \href{https://www.adafruit.com/product/368}{2.1mm Barrel Jack to Screw Terminal Block}
\end{itemize}
% }}}
% Project Strategy {{{
\section{Project Strategy}
    We plan to appropriately wire the joysticks to the Uno, the power supply to the LED matrix, and the LED matrix to the Uno. Then, we hope to design an efficient game engine and implement it in C++. Finally, we hope to 3D print an enclosure for the entire microcontroller assembly such that the two joysticks are in appropriate, comfortable places. 
% }}}
\end{document}
